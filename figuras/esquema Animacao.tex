\begin{tikzpicture}
    \draw [dashdotted]
        % etapa 1
        (0, 0) node {\includegraphics{Lyn1}}
        (0, -1.5) node {\includegraphics{fundo}}
        
        % etapa 2
        (2, 0) node {\includegraphics{Lyn1}}
        (2, -1.5) node {\includegraphics{fundo}}
        ++ (-0.5, 0.5) rectangle ++ (1, -1)
        
        % etapa 3
        (4, 0) node {\includegraphics{Lyn1}}
        (4, -1.5) node {\includegraphics{Lyn2}}
        
        % etapa 4
        (6, 0) node {\includegraphics{fundo}}
        ++ (-0.5, 0.5) rectangle ++ (1, -1)
        (6, -1.5) node {\includegraphics{Lyn2}}
        
        % etapa 5
        (8, 0) node {\includegraphics{fundo}}
        (8, -1.5) node {\includegraphics{Lyn2}}
    ;
    
    \def\eye{
              [fill=white]circle (0.2 and 0.1);
        \fill [gray!85] circle (0.1);
    }
    \draw
        % frames
        (-1, 0) node {\sf\textbf{0}}
        (-1.5, -0.75) --++ (10.5,0)
        (-1, -1.5) node {\sf\textbf{1}}
        
        % legendas
        (0, 0.5) node [above] {I}
        (2, 0.5) node [above] {II}
        (4, 0.5) node [above] {III}
        (6, 0.5) node [above] {IV}
        (8, 0.5) node [above] {V}
    ;
    
    % exibição do frame
    \def\eye{
        arc (30:150:0.2)
        arc (30:150:-0.2)
    }
    \def\pupil{
        circle (0.09)
    }
    \draw
        (0.7, .55) \eye
        (2.7, .55) \eye
        (4.7, -0.95) \eye
        (6.7, -0.95) \eye
        (8.7, -0.95) \eye
    ;
    \fill
        (0.53, .55) \pupil
        (2.53, .55) \pupil
        (4.53, -0.95) \pupil
        (6.53, -0.95) \pupil
        (8.53, -0.95) \pupil
    ;
    
    % descrições
    \draw
        (0, -2.5) node {\small\sf
            \begin{minipage}{2cm}\centering
                Função é\\
                chamada
            \end{minipage}
        }
        
        (2, -2.5) node {\small\sf
            \begin{minipage}{2cm}\centering
                Guarda\\
                o fundo
            \end{minipage}
        }
        
        (4, -2.5) node {\small\sf
            \begin{minipage}{2cm}\centering
                Atualiza pose,\\
                troca o frame
            \end{minipage}
        }
        
        (6, -2.5) node {\small\sf
            \begin{minipage}{2cm}\centering
                Imprime\\
                o fundo
            \end{minipage}
        }
        
        (8, -2.5) node {\small\sf
            \begin{minipage}{2cm}\centering
                Retorna
            \end{minipage}
        }
    ;
    
\end{tikzpicture}