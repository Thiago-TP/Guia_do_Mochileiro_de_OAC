\usepackage{stackengine}
\usepackage{scalerel}

\newcommand\dangersign[1][2ex]{%
  \renewcommand\stacktype{L}%
  \scaleto{\stackon[1.3pt]{\color{red}$\triangle$}{\tiny\bfseries !}}{#1}%
}

\newenvironment{remark}{\par\vspace{10pt}\small % Vertical white space above the remark and smaller font size
  \begin{list}{}{
    \leftmargin=35pt % Indentation on the left
    \rightmargin=25pt
  }\item\ignorespaces % Indentation on the right
  \makebox[-2.5pt]{%
    \begin{tikzpicture}[overlay]
      \node[inner sep=2pt,outer sep=0pt] at 
      (-15pt, 0pt)
      {\dangersign[3.5ex]};
    \end{tikzpicture}
  }
  \advance\baselineskip -1pt}{\end{list}\vskip5pt%
} %


%-----------------
%	THEOREM STYLES
%-----------------
\definecolor{azulunb}{cmyk}{1, 0.65, 0, 0.35}
\definecolor{verdeunb}{cmyk}{ 1, 0, 1, 0.2}

% Boxed/framed environments
\newtheoremstyle{bluenumbox}% Theorem style name
	{0pt}% Space above
	{0pt}% Space below
	{\normalfont}% Body font
	{}% Indent amount
	{\small\bf\sffamily\color{azulunb}}% Theorem head font
	{\;}% Punctuation after theorem head
	{0.25em}% Space after theorem head
	{\small\sffamily\color{azulunb}\thmname{#1}\nobreakspace\thmnumber{#2}\thmnote{\nobreakspace\sffamily\bfseries\color{black}---\nobreakspace#3.}} 
	% Optional theorem note

\newtheoremstyle{blacknumex}% Theorem style name
	{5pt}% Space above
	{5pt}% Space below
	{\normalfont}% Body font
	{} % Indent amount
	{\small\bf\sffamily\color{verdeunb}}% Theorem head font
	{\;}% Punctuation after theorem head
	{0.25em}% Space after theorem head
	{\small\sffamily{\tiny\ensuremath{\blacksquare}}%
		\nobreakspace\thmname{#1}\nobreakspace
		\thmnumber{#2}% Theorem text (e.g. Theorem 2.1)
		\thmnote{\nobreakspace\sffamily\bfseries---\nobreakspace#3.}
	}% Optional theorem note
\makeatother

% Defines the theorem text style for each type of theorem to one of the two styles above

\theoremstyle{bluenumbox}
	\newtheorem{exerciseT}{Exercício}[section]

\theoremstyle{blacknumex}
	\newtheorem{exampleT}{Exemplo}[section]

%------------------------------
%	DEFINITION OF COLORED BOXES
%------------------------------

\RequirePackage[framemethod=default]{mdframed} 
% Required for creating the theorem, definition, exercise and corollary boxes

% Exercise box	  
\newmdenv[
	skipabove=7pt,
	skipbelow=7pt,
	rightline=false,
	leftline=true,
	topline=false,
	bottomline=false,
	backgroundcolor=azulunb!10,
	linecolor=azulunb,
	innerleftmargin=5pt,
	innerrightmargin=5pt,
	innertopmargin=5pt,
	innerbottommargin=5pt,
	leftmargin=0cm,
	rightmargin=0cm,
	linewidth=4pt]{eBox}	

% Creates an environment for each type of theorem and assigns it a theorem text style from the "Theorem Styles" section above and a colored box from above
\newenvironment{exercicio}
    {\begin{eBox}\begin{exerciseT}}{
	    \hfill{\color{azulunb}\tiny\ensuremath{\blacksquare}}
	    \end{exerciseT}\end{eBox}
    }	
%
\newenvironment{exemplo}
    {\begin{exampleT}}
    {\hfill{\color{verdeunb}\tiny\ensuremath{\blacksquare}}\end{exampleT}}	
%


% As linhas abaixo tiram os bad breaks dos warnings. Comente-as se quiser saber se e onde os bad breaks ocorrem
\usepackage{silence}
  \WarningFilter{mdframed}{You got a bad break}
  \makeatletter
  \mdf@PackageWarning{You got a bad break\MessageBreak
    because the last split box is empty\MessageBreak
    You have to change the settings}
  \makeatother
%